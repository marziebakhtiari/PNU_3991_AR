% Generated by GrindEQ Word-to-LaTeX 
\documentclass{article} % use \documentstyle for old LaTeX compilers

\usepackage[english]{babel} % 'french', 'german', 'spanish', 'danish', etc.
\usepackage{amssymb}
\usepackage{amsmath}
\usepackage{txfonts}
\usepackage{mathdots}
\usepackage[classicReIm]{kpfonts}
\usepackage{graphicx}

% You can include more LaTeX packages here 


\begin{document}

%\selectlanguage{english} % remove comment delimiter ('%') and select language if required


\noindent 1. Find the languages generated by the following grammars.

\noindent  a)   S $\mathrm{\to}$ aSa/aba                              b)   S $\mathrm{\to}$ aSb/aAb , A$\mathrm{\to}$ bAa/ba      

\noindent  c) S $\mathrm{\to}$ 0S1/0A1, A $\mathrm{\to}$ 1A/1                d) S $\mathrm{\to}$ 0A/0/1B, A $\mathrm{\to}$ 1A/1, B $\mathrm{\to}$ 0B/1S 

\noindent \textbf{Solution:}

\begin{enumerate}
\item \textbf{ }S is replaced by aSa or aba. aba is a string of terminals. S in aSa can again be replaced by aSa or aba. By this process, the language is
\end{enumerate}

\noindent                                                             S $\mathrm{\to}$ aSa $\mathrm{\to}$ aaSaa {\dots}{\dots}. $\mathrm{\to}$ a(n--1) Sa(n--1) $\mathrm{\to}$ anban 

\noindent                                                               L(G) = anban, n $\mathrm{>}$ 0

\begin{enumerate}
\item   S can be replaced by aSa or aAb. S$\mathrm{\to}$ aAb shifts the control from S to A. Replacing S (n--1) times produces an$\mathrm{-}$1Sbb$\mathrm{-}$1. Replacing S by aAb produces a${}^{n}$Ab${}^{n}$.
\end{enumerate}

\noindent    A is replaced by bAa or ba. As S and A are separate non-terminals, there is no guarantee of replacing S and A at the same time. Let A be replaced (m--1) times, and then ? nally replaced by ba. This generates the string a${}^{n}$b${}^{m}$a${}^{m}$b${}^{n}$. 

\noindent S $\mathrm{\to}$ aSb $\mathrm{\to}$ aaSbb {\dots}. $\mathrm{\to}$ a${}^{n\mathrm{-}1}$Sbb${}^{\mathrm{-}}$${}^{1}$ $\mathrm{\to}$ anAbn $\mathrm{\to}$ a${}^{n}$bAab${}^{n}$ {\dots}. $\mathrm{\to}$ a${}^{n}$b${}^{m\mathrm{-}1}$ A a${}^{m\mathrm{-}1}$bn $\mathrm{\to}$ a${}^{n}$b${}^{m}$a${}^{m}$b${}^{n}$

\noindent  The language generated by the grammar is L(G) = a${}^{n}$b${}^{m}$a${}^{m}$b${}^{n}$, where m, n $\mathrm{>}$ 0. 

\begin{enumerate}
\item  S $\mathrm{\to}$ 0S1 $\mathrm{\to}$ 00S11 {\dots}$\mathrm{\to}$ 0${}^{m\mathrm{-}1}$A1${}^{m\mathrm{-}1\ }$$\mathrm{\to}$ 0${}^{m}$A1${}^{m\ }$$\mathrm{\to}$ 0${}^{m}$1A1${}^{m\ }${\dots}$\mathrm{\to}$ 0${}^{m}$1${}^{n\mathrm{-}1}$A1${}^{m\ }$$\mathrm{\to}$ 0${}^{m}$1${}^{n}$1${}^{m}$ $\mathrm{\to}$ 0${}^{m}$1${}^{m+n\ }$
\end{enumerate}

\noindent The language generated by the grammar is L(G) = 0${}^{m}$1${}^{m+n\ }$where m, n $\mathrm{>}$ 0.

\begin{enumerate}
\item  First, let us try to ? nd the expressions generated by B $\mathrm{\to}$ 0B/1S and A $\mathrm{\to}$ 1A/1. Then, replace the expressions generated by A and B into S $\mathrm{\to}$ 0A/1B.
\end{enumerate}

\noindent 2. Generate a CFG for the language L = Alternating sequence of `a' and `b'.

\noindent Solution: The language may start with `a' or start with `b'. If the string starts with `a', then the language is (ab)${}^{n}$. If the string starts with `b', then the language is (ba)${}^{n}$. If the string starts with `a', then the production rule is S $\mathrm{\to}$ aA . If the string starts with `b', then the production rule is S $\mathrm{\to}$ bB . For alternating sequences of `a' and `b', the production rules are A $\mathrm{\to}$ bB, B $\mathrm{\to}$ aA , A $\mathrm{\to}$b, B $\mathrm{\to}$a.

\noindent The ? nal grammar is

\noindent S $\mathrm{\to}$ aA/bB

\noindent A $\mathrm{\to}$ bB/b

\noindent B$\mathrm{\to}$aA/a

\noindent 3. Find the grammar for the language L = $\mathrm{\{}$a${}^{n}$b${}^{m}$, where n + m is even $\mathrm{\}}$                    [UPTU 2003] \textbf{Solution}: This may happen in three cases if

\noindent   i) both `a' and `b' are odd

\noindent ii) both `a' and `b' are even

\noindent  iii) any one of `a' and `b' are even and the other is zero.

\noindent 

\noindent 

\noindent If the ?rst two are constructed, then the third will be ful?lled. (0 can be considered as even.)

\noindent  The grammar is

\noindent S $\mathrm{\to}$ $\mathrm{\{}$odd number of `a'$\mathrm{\}}$ S $\mathrm{\{}$odd number of `b'$\mathrm{\}}$/

\noindent $\mathrm{\{}$even number of `a'$\mathrm{\}}$ S $\mathrm{\{}$ even number of `b'$\mathrm{\}}$/$\varepsilonup$

\noindent  It can be constructed another way by taking two productions from S; among them, one generates

\noindent both `a' and `b' odd and another generates both `a' and `b' even.

\noindent S$\mathrm{\to}$ AB/CD

\noindent  A $\mathrm{\to}$ aaA/a 

\noindent B $\mathrm{\to}$ bbB/b 

\noindent C $\mathrm{\to}$ aaA/$\varepsilonup$

\noindent D $\mathrm{\to}$ bbB/$\varepsilonup$ 

\noindent 4. Find the grammar for the following language                                                              [UPTU 2004] 

\noindent L = anbnck, where k $\mathrm{\ge}$ 3

\noindent \textbf{Solution:} In the language, there is at least three `c'. The number of `a' and number of `b' are the 

\noindent same and may be null. 

\noindent The grammar is

\noindent S $\mathrm{\to}$ AcccB 

\noindent A $\mathrm{\to}$ aAb/$\varepsilonup$

\noindent B $\mathrm{\to}$ cB/$\varepsilonup$ 

\noindent 5. Construct a grammar which generates all odd integers up to 999.

\noindent \textbf{Solution:} The odd numbers from 0 to 9 are 1, 3, 5, 7, 9. The grammar for generating 1, 3, 5, 7, 9

\noindent  are S $\mathrm{\to}$ 1/3/5/7/9.

\noindent For an odd number of length 2, the grammar is S $\mathrm{\to}$ AS1, where A $\mathrm{\to}$ 0/1{\dots}{\dots}..7/8/9, S1$\mathrm{\to}$ 1/3/5/7/9.

\noindent For an odd number of length 3, the grammar is S $\mathrm{\to}$ BAS1, where A $\mathrm{\to}$ 0/1{\dots}{\dots}..7/8/9, B $\mathrm{\to}$ 0/1{\dots}{\dots}..7/8/9 S1$\mathrm{\to}$ 1/3/5/7/9.

\noindent Combining all these rules, the grammar becomes

\noindent S $\mathrm{\to}$ AS1/BAS1/1/3/5/7/9

\noindent  A $\mathrm{\to}$ 0/1/2/3/4/5/6/7/8/9 

\noindent S ${}_{1}$ $\mathrm{\to}$ 1/3

\noindent 6. Construct a grammar which generates $\mathrm{\{}$\eqref{GrindEQ__01_}${}^{n}$ $\mathrm{\cup }$ \eqref{GrindEQ__10_}${}^{n}$$\mathrm{\}}$, where n $\mathrm{>}$ 0 Solution: It consists of two languages \eqref{GrindEQ__01_}${}^{n}$ and \eqref{GrindEQ__10_}${}^{n}$. Both are connected by union. Take A and B as two non-terminals, which generate \eqref{GrindEQ__01_}${}^{n}$ and \eqref{GrindEQ__10_}${}^{n}$, respectively. From the start symbol S it goes to A and B. The grammar is

\noindent S $\mathrm{\to}$ A/B

\noindent A $\mathrm{\to}$ abA/ab

\noindent  B $\mathrm{\to}$ baB/ba

\noindent  7. Find the grammar for the language L = $\mathrm{\{}$a${}^{n}$b${}^{m}$, where n $\mathrm{\neq}$ m$\mathrm{\}}$. 

\noindent \textbf{Solution:} Here, two cases are possible (i) The number of `a' is more than the number of `b' 

\noindent (ii) The number of `b' is more than the number of`a'.

\noindent For case (i) the grammar is                                A $\mathrm{\to}$ X1Y1 

\noindent X 1$\mathrm{\to}$ aX1/a 

\noindent Y 1 $\mathrm{\to}$ aY1b/$\varepsilonup$

\noindent  For case (ii) the grammar is                             B $\mathrm{\to}$ X2Y2 

\noindent Y 2$\mathrm{\to}$ bY2/b 

\noindent X 2 $\mathrm{\to}$ aX2b/$\varepsilonup$

\noindent The complete grammar is                                 S $\mathrm{\to}$ A/B A $\mathrm{\to}$ X1Y1

\noindent X 1$\mathrm{\to}$ aX1/a 

\noindent Y 1 $\mathrm{\to}$ aY1b/$\varepsilonup$

\noindent  B $\mathrm{\to}$ X2Y2

\noindent  Y 2$\mathrm{\to}$ bY2/b

\noindent X 2 $\mathrm{\to}$ aX2b/$\varepsilonup$

\noindent 8. Find a grammar generating L = $\mathrm{\{}$a${}^{n}$b${}^{n}$cf {\textbar} n $\mathrm{\ge}$ 1, f $\mathrm{\ge}$ 0 $\mathrm{\}}$.                                               [WBUT 2010(IT)] \textbf{Solution}: The language consists of two parts: a${}^{n}$b${}^{n}$, where n $\mathrm{\ge}$ 1, and cf, where f $\mathrm{\ge}$ 0. So, from the start symbol S, we need to take two non-terminals A and B, where A generates a${}^{n}$b${}^{n}$ with n $\mathrm{\ge}$ 1 and B generates cf with f $\mathrm{\ge}$ 0. 

\noindent The grammar for the language is

\noindent S $\mathrm{\to}$ AB

\noindent  A $\mathrm{\to}$ aAb/ab 

\noindent B $\mathrm{\to}$ Bc/$\varepsilonup$ 

\noindent 9. Construct a grammar for the language 

\noindent L = (0 + 1)*111(0 + 1)*

\noindent \textbf{Solution: }It can be thought of as a language set having any combination of 0 or 1 in both the sides of 111. We know that the grammar for \textbf{(0 + 1)}* is A $\mathrm{\to}$ 0A/1A/0/1. 

\noindent From the previous discussion, the grammar is 

\noindent S $\mathrm{\to}$ A111A 

\noindent A $\mathrm{\to}$ 0A/1A/0/1 10

\noindent . Find the grammar for   L - $\mathrm{\{}$a${}^{n}$b${}^{m}$c${}^{m}$ {\textbar} n $\mathrm{\ge}$ 0, m $\mathrm{\ge}$ 1 $\mathrm{\}}$.                                                         [WBUT 2003]

\noindent \textbf{Solution:} There are two parts of the language, a${}^{n\ }$a${}^{n}$d \textbf{b${}^{m}$c${}^{m}$}. Take two non-terminals A and B to generate these two parts.

\noindent S $\mathrm{\to}$ AB A $\mathrm{\to}$ aA/$\varepsilonup$ 

\noindent B $\mathrm{\to}$ bBc/bc 

\noindent 11. Find the grammar for the language L = $\mathrm{\{}$a${}^{n}$b${}^{m\ }${\textbar} n + m is even$\mathrm{\}}$.                                     [UPTU 2004]

\noindent \textbf{Solution:} There are two conditions for n + m to become even.

\noindent  a) both n and m are even 

\noindent b) both n and n are odd

\noindent For case (a), the language is (aa)*(bb)*. For case (b), the language is a(aa)*(bb)*b. For case (a), the grammar is

\noindent S ${}_{1}$ $\mathrm{\to}$ aaS${}_{1}$bb/$\varepsilonup$

\noindent  For case (b), the grammar is

\noindent S ${}_{2}$ $\mathrm{\to}$ aS${}_{1}$b

\noindent  S ${}_{1}$ $\mathrm{\to}$ aaS${}_{1}$bb

\noindent Thus, the ?nal grammar is

\noindent S $\mathrm{\to}$ S1/S2

\noindent  S 1 $\mathrm{\to}$ aaS1bb/$\varepsilonup$

\noindent  S 2 $\mathrm{\to}$ aS1b

\noindent  12. Construct a grammar for the language (i) ai b2 i for i $\mathrm{>}$ 0 and (ii) anban for n $\mathrm{>}$ 0. 

\noindent  [Anna University 06] 

\noindent \textbf{Solution:}

\begin{enumerate}
\item \textbf{ }For a single `a', two `b's are added. The grammar is 
\end{enumerate}

\noindent S $\mathrm{\to}$ aSbb/abb

\noindent  Or

\noindent  S $\mathrm{\to}$ aSbb/aAbb

\noindent  A $\mathrm{\to}$ $\varepsilonup$

\begin{enumerate}
\item  In the language, there is only one `b'. The grammar is
\end{enumerate}

\noindent S $\mathrm{\to}$ aSa/aAa 

\noindent A $\mathrm{\to}$ b

\noindent 

\noindent 

\noindent 

\noindent 

\noindent 

\noindent \textbf{Multiple Choice Questions}

\noindent 

\noindent 

\noindent 

\noindent 1. Which is correct                                                                 

\noindent  a) a+ = a*. a*    b) a* = a+. a+ 

\noindent c) a+ = a*. a       d) a* = a+. a*

\noindent  2. (a, b)* means 

\noindent a) Any combination of a, b including null 

\noindent b) Any combination of a, b excluding null 

\noindent c) Any combination of a, b, but `a' will come ?rst 

\noindent d) None of these 

\noindent 3. (a, b)+ means

\noindent  a) Any combination of a, b including null 

\noindent b) Any combination of a, b excluding null 

\noindent c) Any combination of a, b, but `a' will come ?rst

\noindent  d) None of these

\noindent 

\noindent 

\noindent 

\noindent 

\noindent 

\noindent 

\noindent 

\noindent 

\noindent 

\noindent 

\noindent 

\noindent 

\noindent 

\noindent 4. What is the language generated by the grammar S $\mathrm{\to}$ aSb, S $\mathrm{\to}$ A, A $\mathrm{\to}$ aA

\noindent  a) a${}^{m}$b${}^{m\ }$     b) $\mathrm{\emptyset }$     c) a${}^{n}$b${}^{m}$    d) a${}^{m}$b${}^{m}$ 

\noindent 5. Which type of grammar is the following in particular S $\mathrm{\to}$ aSb S $\mathrm{\to}$ab

\noindent  a) Unrestricted    

\noindent b) Context-sensitive grammar

\noindent  c) Context-free grammar

\noindent  d) Regular grammar 

\noindent 6. Which type of grammar is the following in particular S $\mathrm{\to}$ aS/bA  A $\mathrm{\to}$ aA/a

\noindent  a) Unrestricted

\noindent  b) Context-sensitive grammar 

\noindent c) Context-free grammar

\noindent  d) Regular grammar


\end{document}

